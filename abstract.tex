\pagestyle{empty}
{\textbf{Abstract}\\[1cm]}

This thesis studies all the stages involved in the recognition, control and manipulation of laparoscopic tools in an end-to-end approach with more emphasis given to the pivot trajectories and the RCM constrained motion 
planning. The first step was to study the forward and inverse kinematics of the KUKA iiwa14 industrial robot arm (with the position-orientation decoupling technique for 6dof robots, leaving the extra degree of freedom to be 
specified by other constraints) as well as kinematics of the Barrett hand gripper in order to calculate grasps. Since MIS robotics is very constrained in nature, the next step was to study the RCM constraint for the pivot 
motions, the elbow-up constraint to avoid collisions as well as the workspace constraints and singularities. When designing robot applications it is important to study the robot’s workspace, and for this reason, this thesis 
also studies the surgical task space and how this is transformed to the robot’s taskspace and the joint space. The manipulability of the robot was also studied and also a suitable environment layout so that all trajectories 
were well reachable. A lot of emphasis was given in calculating various geometric paths for the robot to follow inside the surgical task. The equations for circular, circular arc, line segment, helical, cubic spline, b-spline, 
higher-order polynomial and trapezoid and s-curve velocity profile trajectories are studied in detail in order to generate pivot motions with a wide variety. All experiments were conducted using the ROS framework and popular 
tools and libraries like Gazebo, RViz and MoveIt using the RRTConnect path planning algorithm. The experiments were evaluated with measurements of time, position accuracy and RCM distance deviation. This thesis also briefly 
studies a simple recognition of a laparoscopic tool and the estimation of it’s position and orientation using computer vision as well as the calculation of 3 points on the surgical tool where the gripper’s fingers will be 
placed in order to grasp the object with a satisfactory force closure. Finally this thesis studies some control system schemes like for example the RCM tracking and pivot motion control.

\newpage{\pagestyle{empty}\cleardoublepage}

\newpage
\pagestyle{empty}
{\textbf{Περίληψη}\\[1cm]}

Η παρούσα διπλωματική εξετάζει όλα τα στάδια που εμπλέκονται στην αναγνώριση, τον έλεγχο και τον χειρισμό των λαπαροσκοπικών εργαλείων με μια ολιστική προσέγγιση αλλά με μεγαλύτερη έμφαση να δίνεται στις τροχιές περιστροφής 
γύρω από σημείο και στον προγραμματισμό περιορισμένης κίνησης RCM. Το πρώτο βήμα ήταν να μελετηθεί το ευθύ και αντίστροφο κινηματικό πρόβλημα του βιομηχανικού ρομποτικού βραχίονα KUKA iiwa14 (με την τεχνική αποσύνδεσης θέσης-
προσανατολισμού για ρομπότ 6 βαθμών ελευθερίας, αφήνοντας τον επιπλέον βαθμό ελευθερίας να καθοριστεί από άλλους περιορισμούς) καθώς και η κινηματική της αρπάγης Barrett ώστε να υπολογιστούν λαβές του χειρουργικού εργαλείου. 
Δεδομένου ότι η ρομποτική MIS επιβάλλει πολλούς περιορισμούς, το επόμενο βήμα ήταν να μελετηθεί ο περιορισμός RCM για τις κινήσεις περιστροφής (pivot), o περιορισμός στον οποίο ο αγκώνας του ρομπότ πρέπει να είναι προς τα πάνω 
για την αποφυγή συγκρούσεων καθώς και οι περιορισμοί και τα σημεία ενικότητας του χώρου εργασίας. Κατά το σχεδιασμό εφαρμογών ρομπότ είναι σημαντικό να μελετηθεί ο χώρος εργασίας του ρομπότ και για το λόγο αυτό, αυτή στη 
διπλωματική αυτή μελετάται επιπλέον ο χώρος χειρουργικών εργασιών και πώς αυτός μετατρέπεται στον χώρο εργασιών του ρομπότ και στον χώρο αρθρώσεών του. Μελετήθηκε επίσης ο δείκτης επιδεξιότητας του ρομπότ και η κατάλληλη 
διάταξη του ρομπότ σε σχεση με το περιβάλλον του έτσι ώστε όλες οι τροχιές να είναι καλά προσβάσιμες και να μπορουν να εκτελεστούν με ευκολία. Δόθηκε μεγάλη έμφαση στον υπολογισμό διαφόρων γεωμετρικών μονοπατιών που έπρεπε να 
ακολουθήσει το ρομπότ μέσα στο χώρο χειρουργικής εργασίας. Οι εξισώσεις για τροχιές κύκλου, κυκλικού τόξου, ευθύγραμμου τμήματος, έλικα, κυβικού spline, b-spline, πολυωνύμων υψηλότερης τάξης και τροχιές με προφίλ ταχύτητας 
τραπεζοειδές και καμπύλης-s, μελετώνται λεπτομερώς, προκειμένου να δημιουργηθούν χειρουργικές ρομποτικές κινήσεις με μεγάλη ποικιλία. Όλα τα πειράματα υλοποιήθηκαν χρησιμοποιώντας το περιβάλλον ROS και δημοφιλή εργαλεία και 
βιβλιοθήκες όπως τα Gazebo, RViz και MoveIt χρησιμοποιώντας τον αλγόριθμο σχεδιασμού διαδρομής RRTConnect. Τα πειράματα αξιολογήθηκαν με μετρήσεις χρόνου, ακρίβειας θέσης και απόκλισης απόστασης RCM. Στη διπλωματική αυτή 
μελετάται επίσης εν συντομία η απλή αναγνώριση ενός λαπαροσκοπικού εργαλείου και η εκτίμηση της θέσης και του προσανατολισμού του με χρήση υπολογιστικής όρασης καθώς και ο υπολογισμός 3 σημείων πάνω στο χειρουργικό εργαλείο 
όπου θα τοποθετηθούν τα δάχτυλα της αρπάγης για να πιάσει το αντικείμενο με μία ικανοποιητική λαβή. Τέλος, σε αυτή τη διπλωματική μελετώνται ορισμένα συστήματα ελέγχου όπως για παράδειγμα ενός συστήματος για την παρακολούθηση 
RCM και τον έλεγχο κίνησης περιστροφής.
