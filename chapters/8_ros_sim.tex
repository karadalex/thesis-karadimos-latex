\section{Implementation with the ROS framework}

\subsection{Introduction to the ROS framework}

\subsection{Gazebo simulation environment}

\begin{center}
\begin{figure}[H]
\centering
\includegraphics[width=12cm]{images/gazebo-sim1.png}\\
\caption{Simulation environment in Gazebo}
\end{figure}
\end{center}

The main environment setup of this thesis was designed using the Gazebo simulation environment and 
it consists of the following objects:
\begin{itemize}
\item the robot arm, KUKA\textsuperscript \textregistered iiwa14 lbr, being at the center of the setup
\item the robot base, so that the robot arm can better reach the tools and the surgical site and have more flexibility in movment
\item 2 tables, one for the tools and one for the surgical site
\item 4 surgical tools, using a modified version of the surgical tools used in the Raven II surgical platform
\item a mounting dock, which has holes that have the same role as the trocars (small tubes from 
which the surgical tool is inserted). Initially a mounting dock with 4 same holes of 4mm diameter was used, but it was later replaced with a new one with holes of variable diameters to test feasibility of pivot motions. Larger diameters means more space for motion planner to search for solution and thus more probable to find a solution.
\end{itemize}


\subsection{Visualization and Motion Planning with RViz and Moveit}

Motion Planning parameters outside of body:
\begin{itemize}
	\item Position tolerance: 50μm
	\item Orientation tolerance: 0.00005 deg
	\item Planning time: 10s
\end{itemize}

Motion Planning parameters inside of body:
\begin{itemize}
	\item Position tolerance:
	\item Orientation tolerance:
	\item Planning time
	\item End-effector interpolation step: 1mm
	\item Maximum velocity scaling factor
\end{itemize}

Sometimes the motion planner finds a solution but the execution from the controller is aborted. 
After many iterations of the same experiment this does not happen always, which means that the 
feasibility of the execution of the movement by the controller depends on the initial state of 
the robot, i.e. if initially some joints of the robot are at their boundaries, then the next 
commanded trajectory maybe unfeasible.

At each time step it is important to publish a custom message containing all the information 
about the kinematic state of the robot. In this thesis a custom \textbf{ROS} message was created 
containing a tf transform with a 3D vector for the position and a quaternion for the rotation and 
a custom  6-by-7 matrix containing the values of the Jacobian. The MoveIt library, from which the 
kinematic state of the robot is obtained, returns the orientation of the end effector as a 3-by-3 
rotation matrix, but in the ROS tf message it must be expressed as a quaternion. To convert the 
matrix to a quaternion we first calculate the euler angles and then use these values to construct 
the quaternion “vector”. The quaternion representation of rotation is often preferred in robotic 
applications due to its efficiency in calculations and memory. To convert the transformation 
matrix to euler angles and then to quaternions the following formulas were used:
\[
T = 
\begin{bmatrix}
r_{11} & r_{12} & r_{13} & x \\
r_{21} & r_{22} & r_{23} & y \\
r_{31} & r_{32} & r_{33} & z \\
0 & 0 & 0 & 1\\
\end{bmatrix}
\]

\[
φ = atan2(r_{21}, r_{11})
\]

\[
θ = atan2(-r_{31}, \sqrt{r_{11}^2 + r_{21}^2})
\]

\[
ψ = atan2(r_{32}, r_{33})
\]

where $T$ is the transformation matrix and $φ, θ, ψ$ are the roll, pitch and yaw (Euler) angles.




\subsection{Experiments and Development methodology}

\subsubsection{Robot Planner 1}

\subsubsection{Robot Planner 2}

\subsubsection{Robot Planner 3}

\subsubsection{Robot Planner 4}

\subsubsection{Robot Planner 5}

